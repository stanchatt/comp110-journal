% Please do not change the document class
\documentclass{scrartcl}

% Please do not change these packages
\usepackage[hidelinks]{hyperref}
\usepackage[none]{hyphenat}
\usepackage{setspace}
\doublespace

% Please include a clear, concise, and descriptive title
\title{Research essay}

% Please do not change the subtitle
\subtitle{COMP150 - Agile Development Practice}

% Please put your student number in the author field
\author{1706194}

\begin{document}
	
	\maketitle
	

	
	\section{Illumination for computer generated pictures}	
The research article I have chosen is "Illumination for computer generated pictures" by T. Phong.\\ 
The paper begins by stating that computers can generate images of solid objects, fully shaded, reasonably fast due to the number of different algorithms at the computers disposal. However, despite the speed of images being generated, they will never appear as realistic as we would like them to be due to textures, lighting, and shadows like a photograph could capture. The paper also mentions that a computer-generated image is achieved by a numerical model that’s stored as an objective description. The paper uses a metaphor to help visualise this, "The computer system can be compared to an artist who paints an object from it description and not from direct observation of the object". This is why the computer-generated images never look as realistic, because the computer has never seen the objects in real life. 
	
	\section{Example of shading techniques}
	
	The paper then goes on to talk about different shading techniques and algorithms a computer uses to shade in solid objects, for example: 
	\begin{itemize}
	\item Warnocks shading: Warnock shading is used if any objects have lost their depth and appear flat. It restores this by decreasing the intensity of any light reflection given off by the object with the distance of the light source.\\  
	Warnock places the light source and the eye in the same position, this means that the shading function would be the sum of two terms. The overall pictures would have many attributes, like identical parallel faces which would be shaded with different intensities. Any faces that are facing the light source directly would be the brightest, and faces next to it would have low intensity of shading which would increase as you move around the shape. 
	The problem with this is that the shape loses its smoothness.\\  
	\item Newell, Newell and Sanchas shading: This shading technique mostly uses highlights and transparency after Newell, Newell and Sanchas noticed that in the real-world highlights aren't just created by the light source itself but also the reflections from other objects surrounding it. This technique is used for making curved objects appear to be made from glass, giving it transparency and highlights from light sources.
\end{itemize} 
	
	\section{Reason of choice}
	The reason why I chose "Illumination for computer generated pictures" paper over \cite{one} paper was because I felt that the paper just wasn’t gripping enough for me. I felt like the paper was lacking any interesting content, compared to the "Illumination for computer generated pictures" paper which talks about Phong shading and how its implemented in Computer graphics. I feel like the "Illumination for computer generated pictures" paper has more practical uses compared to \cite{one} because the paper talks about so many different techniques and methods, which use different algorithms, to achieve different effects.\\
	
	The reason why I chose this paper over \cite{two} was because I struggled to comprehend the mathematic equations in the paper. While I was reading the paper, I had to keep rereading and go over the same equations but I just felt lost when it came to understanding what it was trying to explain. I feel like you would need a basic understanding to read the math's that are included with this is paper to fully appreciate its content. The "Illumination for computer generated pictures" paper was so much easier for me to get my head around and understand its content. This is because of how the paper is written and structured. I had to keep rereading \cite{two} equations because they felt out of place and ruined the flow for me.\\
	
	I didn’t choose \cite{three} because its like a combination of \cite{one} and \cite{two}. I found the papers context not as interesting compared to ""Illumination for computer generated pictures". This is because \cite{three} context was about how computers make a decision using certain algorithms. \cite{three} uses a good example, explaining how the AI makes its decision in a game of chess using the algorithm. This example did make things easier to understand, however the diagrams in ""Illumination for computer generated pictures" made it easier for me to understand how the different shading techniques worked and how the images looked after they were generated. I also found some of the pseudo code confusing and hard to follow. This meant I had similar problems like with \cite{two}, because I found the pseudo code hard to follow, I had to keep rereading through the code in an attempt to understand what is was explaining.\\
	
	The only paper that was a close second for my choice was \cite{four}. This papers topic and context was very interesting, asking if a computer had the ability to think and make a decision based on thoughts and information presented to it. The paper changed its question very early in the paper to make things clearer considering you could have many of definitions of what could define as a "computer" "and thinking", so the paper used a very good example to help example called the "imitation game" to explain how the computer can make a decision based on certain answers. However, the reason why I chose the “Ilumination for computer generated pictures" paper was because I found that the diagrams that were given gave more insight to what the paper was describing compared to the game example in \cite{four}.\\
	      
	\section{Conclusion}
	
Overall, I found that “Ilumination for computer generated pictures" paper to be the most insightful and more interesting paper due to its explanations, examples and wide variety of techniques that are used in computer graphics when it comes to shading in objects. The diagrams were clear and gave a brilliant visualisation of what the different shading techniques can achieve. The maths equations were easier to understand compared to some other papers like \cite{two} \cite{three} which meant I could easily understand how the shading algorithms worked without having to go over the same equations multiple times.
  
	\bibliographystyle{ieeetran}
	\bibliography{references}

		
	
\end{document}